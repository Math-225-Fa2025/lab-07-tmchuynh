% Options for packages loaded elsewhere
\PassOptionsToPackage{unicode}{hyperref}
\PassOptionsToPackage{hyphens}{url}
\documentclass[
]{article}
\usepackage{xcolor}
\usepackage[margin=1in]{geometry}
\usepackage{amsmath,amssymb}
\setcounter{secnumdepth}{5}
\usepackage{iftex}
\ifPDFTeX
  \usepackage[T1]{fontenc}
  \usepackage[utf8]{inputenc}
  \usepackage{textcomp} % provide euro and other symbols
\else % if luatex or xetex
  \usepackage{unicode-math} % this also loads fontspec
  \defaultfontfeatures{Scale=MatchLowercase}
  \defaultfontfeatures[\rmfamily]{Ligatures=TeX,Scale=1}
\fi
\usepackage{lmodern}
\ifPDFTeX\else
  % xetex/luatex font selection
\fi
% Use upquote if available, for straight quotes in verbatim environments
\IfFileExists{upquote.sty}{\usepackage{upquote}}{}
\IfFileExists{microtype.sty}{% use microtype if available
  \usepackage[]{microtype}
  \UseMicrotypeSet[protrusion]{basicmath} % disable protrusion for tt fonts
}{}
\makeatletter
\@ifundefined{KOMAClassName}{% if non-KOMA class
  \IfFileExists{parskip.sty}{%
    \usepackage{parskip}
  }{% else
    \setlength{\parindent}{0pt}
    \setlength{\parskip}{6pt plus 2pt minus 1pt}}
}{% if KOMA class
  \KOMAoptions{parskip=half}}
\makeatother
\usepackage{color}
\usepackage{fancyvrb}
\newcommand{\VerbBar}{|}
\newcommand{\VERB}{\Verb[commandchars=\\\{\}]}
\DefineVerbatimEnvironment{Highlighting}{Verbatim}{commandchars=\\\{\}}
% Add ',fontsize=\small' for more characters per line
\usepackage{framed}
\definecolor{shadecolor}{RGB}{248,248,248}
\newenvironment{Shaded}{\begin{snugshade}}{\end{snugshade}}
\newcommand{\AlertTok}[1]{\textcolor[rgb]{0.94,0.16,0.16}{#1}}
\newcommand{\AnnotationTok}[1]{\textcolor[rgb]{0.56,0.35,0.01}{\textbf{\textit{#1}}}}
\newcommand{\AttributeTok}[1]{\textcolor[rgb]{0.13,0.29,0.53}{#1}}
\newcommand{\BaseNTok}[1]{\textcolor[rgb]{0.00,0.00,0.81}{#1}}
\newcommand{\BuiltInTok}[1]{#1}
\newcommand{\CharTok}[1]{\textcolor[rgb]{0.31,0.60,0.02}{#1}}
\newcommand{\CommentTok}[1]{\textcolor[rgb]{0.56,0.35,0.01}{\textit{#1}}}
\newcommand{\CommentVarTok}[1]{\textcolor[rgb]{0.56,0.35,0.01}{\textbf{\textit{#1}}}}
\newcommand{\ConstantTok}[1]{\textcolor[rgb]{0.56,0.35,0.01}{#1}}
\newcommand{\ControlFlowTok}[1]{\textcolor[rgb]{0.13,0.29,0.53}{\textbf{#1}}}
\newcommand{\DataTypeTok}[1]{\textcolor[rgb]{0.13,0.29,0.53}{#1}}
\newcommand{\DecValTok}[1]{\textcolor[rgb]{0.00,0.00,0.81}{#1}}
\newcommand{\DocumentationTok}[1]{\textcolor[rgb]{0.56,0.35,0.01}{\textbf{\textit{#1}}}}
\newcommand{\ErrorTok}[1]{\textcolor[rgb]{0.64,0.00,0.00}{\textbf{#1}}}
\newcommand{\ExtensionTok}[1]{#1}
\newcommand{\FloatTok}[1]{\textcolor[rgb]{0.00,0.00,0.81}{#1}}
\newcommand{\FunctionTok}[1]{\textcolor[rgb]{0.13,0.29,0.53}{\textbf{#1}}}
\newcommand{\ImportTok}[1]{#1}
\newcommand{\InformationTok}[1]{\textcolor[rgb]{0.56,0.35,0.01}{\textbf{\textit{#1}}}}
\newcommand{\KeywordTok}[1]{\textcolor[rgb]{0.13,0.29,0.53}{\textbf{#1}}}
\newcommand{\NormalTok}[1]{#1}
\newcommand{\OperatorTok}[1]{\textcolor[rgb]{0.81,0.36,0.00}{\textbf{#1}}}
\newcommand{\OtherTok}[1]{\textcolor[rgb]{0.56,0.35,0.01}{#1}}
\newcommand{\PreprocessorTok}[1]{\textcolor[rgb]{0.56,0.35,0.01}{\textit{#1}}}
\newcommand{\RegionMarkerTok}[1]{#1}
\newcommand{\SpecialCharTok}[1]{\textcolor[rgb]{0.81,0.36,0.00}{\textbf{#1}}}
\newcommand{\SpecialStringTok}[1]{\textcolor[rgb]{0.31,0.60,0.02}{#1}}
\newcommand{\StringTok}[1]{\textcolor[rgb]{0.31,0.60,0.02}{#1}}
\newcommand{\VariableTok}[1]{\textcolor[rgb]{0.00,0.00,0.00}{#1}}
\newcommand{\VerbatimStringTok}[1]{\textcolor[rgb]{0.31,0.60,0.02}{#1}}
\newcommand{\WarningTok}[1]{\textcolor[rgb]{0.56,0.35,0.01}{\textbf{\textit{#1}}}}
\usepackage{graphicx}
\makeatletter
\newsavebox\pandoc@box
\newcommand*\pandocbounded[1]{% scales image to fit in text height/width
  \sbox\pandoc@box{#1}%
  \Gscale@div\@tempa{\textheight}{\dimexpr\ht\pandoc@box+\dp\pandoc@box\relax}%
  \Gscale@div\@tempb{\linewidth}{\wd\pandoc@box}%
  \ifdim\@tempb\p@<\@tempa\p@\let\@tempa\@tempb\fi% select the smaller of both
  \ifdim\@tempa\p@<\p@\scalebox{\@tempa}{\usebox\pandoc@box}%
  \else\usebox{\pandoc@box}%
  \fi%
}
% Set default figure placement to htbp
\def\fps@figure{htbp}
\makeatother
\setlength{\emergencystretch}{3em} % prevent overfull lines
\providecommand{\tightlist}{%
  \setlength{\itemsep}{0pt}\setlength{\parskip}{0pt}}
\usepackage[]{natbib}
\bibliographystyle{plainnat}
\usepackage{bookmark}
\IfFileExists{xurl.sty}{\usepackage{xurl}}{} % add URL line breaks if available
\urlstyle{same}
\hypersetup{
  pdftitle={Lab 07 - Smokers in Whickham},
  pdfauthor={Tina Huynh},
  hidelinks,
  pdfcreator={LaTeX via pandoc}}

\title{Lab 07 - Smokers in Whickham}
\usepackage{etoolbox}
\makeatletter
\providecommand{\subtitle}[1]{% add subtitle to \maketitle
  \apptocmd{\@title}{\par {\large #1 \par}}{}{}
}
\makeatother
\subtitle{Simpson's paradox}
\author{Tina Huynh}
\date{}

\begin{document}
\maketitle

{
\setcounter{tocdepth}{2}
\tableofcontents
}
\pandocbounded{\includegraphics[keepaspectratio]{img/whickham.png}}

A study of conducted in Whickham, England recorded participants' age,
smoking status at baseline, and then 20 years later recorded their
health outcome. In this lab we analyse the relationships between these
variables, first two at a time, and then controlling for the third.

\section{Learning goals}\label{learning-goals}

\begin{itemize}
\tightlist
\item
  Visualising relationships between variables
\item
  Discovering Simpson's paradox via visualisations
\end{itemize}

\section{Getting started}\label{getting-started}

Go to the course GitHub organization and locate your homework repo,
clone it in RStudio and open the R Markdown document. Knit the document
to make sure it compiles without errors.

\subsection{Warm up}\label{warm-up}

Before we introduce the data, let's warm up with some simple exercises.
Update the YAML of your R Markdown file with your information, knit,
commit, and push your changes. Make sure to commit with a meaningful
commit message. Then, go to your repo on GitHub and confirm that your
changes are visible in your Rmd \textbf{and} md files. If anything is
missing, commit and push again.

\subsection{Packages}\label{packages}

We'll use the \textbf{tidyverse} package for much of the data wrangling
and visualisation and the data lives in the \textbf{mosaicData} package.
These packages are already installed for you. You can load them by
running the following in your Console:

\begin{Shaded}
\begin{Highlighting}[]
\FunctionTok{library}\NormalTok{(tidyverse)}
\FunctionTok{library}\NormalTok{(mosaicData)}
\end{Highlighting}
\end{Shaded}

\subsection{Data}\label{data}

The dataset we'll use is called Whickham from the \textbf{mosaicData}
package. You can find out more about the dataset by inspecting their
documentation, which you can access by running \texttt{?Whickham} in the
Console or using the Help menu in RStudio to search for
\texttt{Whickham}.

\section{Exercises}\label{exercises}

\begin{enumerate}
\def\labelenumi{\arabic{enumi}.}
\item
  What type of study do you think these data come from: observational or
  experiment? Why? I think these data come from an observational study
  because the researchers did not manipulate any variables, they simply
  observed and recorded the smoking status and health outcomes of the
  participants over a 20-year period.
\item
  How many observations are in this dataset? What does each observation
  represent? There are 5000 observations in this dataset. Each
  observation represents an individual participant in the study, with
  their corresponding age, smoking status at baseline, and health
  outcome after 20 years.
\item
  How many variables are in this dataset? What type of variable is each?
  Display each variable using an appropriate visualization. There are 3
  variables in this dataset:
\end{enumerate}

\begin{itemize}
\tightlist
\item
  Age: Continuous variable (numeric)
\item
  Smoker: Categorical variable (factor with levels ``yes'' and ``no'')
\item
  Outcome: Categorical variable (factor with levels ``alive'' and
  ``dead'')
\end{itemize}

\begin{Shaded}
\begin{Highlighting}[]
\CommentTok{\# Visualizing Age}
\FunctionTok{ggplot}\NormalTok{(Whickham, }\FunctionTok{aes}\NormalTok{(}\AttributeTok{x =}\NormalTok{ age)) }\SpecialCharTok{+}
  \FunctionTok{geom\_histogram}\NormalTok{(}\AttributeTok{binwidth =} \DecValTok{5}\NormalTok{, }\AttributeTok{fill =} \StringTok{"blue"}\NormalTok{, }\AttributeTok{color =} \StringTok{"black"}\NormalTok{) }\SpecialCharTok{+}
  \FunctionTok{labs}\NormalTok{(}\AttributeTok{title =} \StringTok{"Distribution of Age"}\NormalTok{, }\AttributeTok{x =} \StringTok{"Age"}\NormalTok{, }\AttributeTok{y =} \StringTok{"Count"}\NormalTok{)}
\end{Highlighting}
\end{Shaded}

\begin{Shaded}
\begin{Highlighting}[]
\CommentTok{\# Visualizing Smoker}
\FunctionTok{ggplot}\NormalTok{(Whickham, }\FunctionTok{aes}\NormalTok{(}\AttributeTok{x =}\NormalTok{ smoker)) }\SpecialCharTok{+}
  \FunctionTok{geom\_bar}\NormalTok{(}\AttributeTok{fill =} \StringTok{"green"}\NormalTok{, }\AttributeTok{color =} \StringTok{"black"}\NormalTok{) }\SpecialCharTok{+}
  \FunctionTok{labs}\NormalTok{(}\AttributeTok{title =} \StringTok{"Smoking Status"}\NormalTok{, }\AttributeTok{x =} \StringTok{"Smoker"}\NormalTok{, }\AttributeTok{y =} \StringTok{"Count"}\NormalTok{)}
\end{Highlighting}
\end{Shaded}

\begin{Shaded}
\begin{Highlighting}[]
\CommentTok{\# Visualizing Outcome}
\FunctionTok{ggplot}\NormalTok{(Whickham, }\FunctionTok{aes}\NormalTok{(}\AttributeTok{x =}\NormalTok{ outcome)) }\SpecialCharTok{+}
  \FunctionTok{geom\_bar}\NormalTok{(}\AttributeTok{fill =} \StringTok{"red"}\NormalTok{, }\AttributeTok{color =} \StringTok{"black"}\NormalTok{) }\SpecialCharTok{+}
  \FunctionTok{labs}\NormalTok{(}\AttributeTok{title =} \StringTok{"Health Outcome"}\NormalTok{, }\AttributeTok{x =} \StringTok{"Outcome"}\NormalTok{, }\AttributeTok{y =} \StringTok{"Count"}\NormalTok{)}
\end{Highlighting}
\end{Shaded}

\begin{enumerate}
\def\labelenumi{\arabic{enumi}.}
\setcounter{enumi}{3}
\tightlist
\item
  What would you expect the relationship between smoking status and
  health outcome to be? I would expect that smokers are more likely to
  have a negative health outcome (i.e., be dead) compared to
  non-smokers. This expectation is based on the well-established link
  between smoking and various health issues, including respiratory
  diseases, cardiovascular diseases, and cancer, which can lead to
  higher mortality rates among smokers.
\end{enumerate}

Knit, \emph{commit, and push your changes to GitHub with an appropriate
commit message. Make sure to commit and push all changed files so that
your Git pane is cleared up afterwards.}

\begin{enumerate}
\def\labelenumi{\arabic{enumi}.}
\setcounter{enumi}{4}
\tightlist
\item
  Create a visualization depicting the relationship between smoking
  status and health outcome. Briefly describe the relationship, and
  evaluate whether this meets your expectations. Additionally, calculate
  the relevant conditional probabilities to help your narrative. Here is
  some code to get you started:
\end{enumerate}

\begin{Shaded}
\begin{Highlighting}[]
\NormalTok{Whickham }\SpecialCharTok{\%\textgreater{}\%}
  \FunctionTok{count}\NormalTok{(smoker, outcome) }\SpecialCharTok{\%\textgreater{}\%}
  \FunctionTok{ggplot}\NormalTok{(}\FunctionTok{aes}\NormalTok{(}\AttributeTok{x =}\NormalTok{ smoker, }\AttributeTok{y =}\NormalTok{ n, }\AttributeTok{fill =}\NormalTok{ outcome)) }\SpecialCharTok{+}
  \FunctionTok{geom\_bar}\NormalTok{(}\AttributeTok{stat =} \StringTok{"identity"}\NormalTok{, }\AttributeTok{position =} \StringTok{"fill"}\NormalTok{) }\SpecialCharTok{+}
  \FunctionTok{labs}\NormalTok{(}\AttributeTok{y =} \StringTok{"Proportion"}\NormalTok{, }\AttributeTok{title =} \StringTok{"Health Outcome by Smoking Status"}\NormalTok{) }\SpecialCharTok{+}
  \FunctionTok{scale\_y\_continuous}\NormalTok{(}\AttributeTok{labels =}\NormalTok{ scales}\SpecialCharTok{::}\NormalTok{percent)}
\end{Highlighting}
\end{Shaded}

\begin{Shaded}
\begin{Highlighting}[]
\CommentTok{\# Calculate conditional probabilities}
\NormalTok{Whickham }\SpecialCharTok{\%\textgreater{}\%}
  \FunctionTok{count}\NormalTok{(smoker, outcome) }\SpecialCharTok{\%\textgreater{}\%}
  \FunctionTok{group\_by}\NormalTok{(smoker) }\SpecialCharTok{\%\textgreater{}\%}
  \FunctionTok{mutate}\NormalTok{(}\AttributeTok{prop =}\NormalTok{ n }\SpecialCharTok{/} \FunctionTok{sum}\NormalTok{(n)) }\SpecialCharTok{\%\textgreater{}\%}
  \FunctionTok{select}\NormalTok{(smoker, outcome, prop) }\SpecialCharTok{\%\textgreater{}\%}
    \FunctionTok{pivot\_wider}\NormalTok{(}\AttributeTok{names\_from =}\NormalTok{ outcome, }\AttributeTok{values\_from =}\NormalTok{ prop)}
\end{Highlighting}
\end{Shaded}

Surprisingly, the visualization shows that smokers have a lower
proportion of deaths (around 24\%) compared to non-smokers (around
31\%). This does NOT meet my expectations, as I would have expected
smokers to have higher mortality rates. This counterintuitive result
suggests there may be a confounding variable at play.

The conditional probabilities show that among smokers, 76.3\% are alive
and 23.7\% are dead, while among non-smokers, 68.6\% are alive and
31.4\% are dead. This means smokers have a lower death rate (23.7\%)
compared to non-smokers (31.4\%). This counterintuitive result suggests
there may be a confounding variable at play.

\begin{enumerate}
\def\labelenumi{\arabic{enumi}.}
\setcounter{enumi}{5}
\tightlist
\item
  Create a new variable called \texttt{age\_cat} using the following
  scheme:
\end{enumerate}

\begin{itemize}
\tightlist
\item
  \texttt{age\ \textless{}=\ 44\ \textasciitilde{}\ "18-44"}
\item
  \texttt{age\ \textgreater{}\ 44\ \&\ age\ \textless{}=\ 64\ \textasciitilde{}\ "45-64"}
\item
  \texttt{age\ \textgreater{}\ 64\ \textasciitilde{}\ "65+"}
\end{itemize}

\begin{Shaded}
\begin{Highlighting}[]
\NormalTok{Whickham }\OtherTok{\textless{}{-}}\NormalTok{ Whickham }\SpecialCharTok{\%\textgreater{}\%}
  \FunctionTok{mutate}\NormalTok{(}\AttributeTok{age\_cat =} \FunctionTok{case\_when}\NormalTok{(}
\NormalTok{    age }\SpecialCharTok{\textless{}=} \DecValTok{44} \SpecialCharTok{\textasciitilde{}} \StringTok{"18{-}44"}\NormalTok{,}
\NormalTok{    age }\SpecialCharTok{\textgreater{}} \DecValTok{44} \SpecialCharTok{\&}\NormalTok{ age }\SpecialCharTok{\textless{}=} \DecValTok{64} \SpecialCharTok{\textasciitilde{}} \StringTok{"45{-}64"}\NormalTok{,}
\NormalTok{    age }\SpecialCharTok{\textgreater{}} \DecValTok{64} \SpecialCharTok{\textasciitilde{}} \StringTok{"65+"}
\NormalTok{  )) }\SpecialCharTok{\%\textgreater{}\%}
  \FunctionTok{mutate}\NormalTok{(}\AttributeTok{age\_cat =} \FunctionTok{factor}\NormalTok{(age\_cat, }\AttributeTok{levels =} \FunctionTok{c}\NormalTok{(}\StringTok{"18{-}44"}\NormalTok{, }\StringTok{"45{-}64"}\NormalTok{, }\StringTok{"65+"}\NormalTok{)))}

\FunctionTok{table}\NormalTok{(Whickham}\SpecialCharTok{$}\NormalTok{age\_cat)}
\end{Highlighting}
\end{Shaded}

\begin{enumerate}
\def\labelenumi{\arabic{enumi}.}
\setcounter{enumi}{6}
\tightlist
\item
  Re-create the visualization depicting the relationship between smoking
  status and health outcome, faceted by \texttt{age\_cat}. What changed?
  What might explain this change? Extend the contingency table from
  earlier by breaking it down by age category and use it to help your
  narrative.
\end{enumerate}

\begin{Shaded}
\begin{Highlighting}[]
\NormalTok{Whickham }\SpecialCharTok{\%\textgreater{}\%}
  \FunctionTok{count}\NormalTok{(smoker, age\_cat, outcome) }\SpecialCharTok{\%\textgreater{}\%}
  \FunctionTok{ggplot}\NormalTok{(}\FunctionTok{aes}\NormalTok{(}\AttributeTok{x =}\NormalTok{ smoker, }\AttributeTok{y =}\NormalTok{ n, }\AttributeTok{fill =}\NormalTok{ outcome)) }\SpecialCharTok{+}
  \FunctionTok{geom\_bar}\NormalTok{(}\AttributeTok{stat =} \StringTok{"identity"}\NormalTok{, }\AttributeTok{position =} \StringTok{"fill"}\NormalTok{) }\SpecialCharTok{+}
  \FunctionTok{labs}\NormalTok{(}\AttributeTok{y =} \StringTok{"Proportion"}\NormalTok{, }\AttributeTok{title =} \StringTok{"Health Outcome by Smoking Status and Age Category"}\NormalTok{) }\SpecialCharTok{+}
  \FunctionTok{scale\_y\_continuous}\NormalTok{(}\AttributeTok{labels =}\NormalTok{ scales}\SpecialCharTok{::}\NormalTok{percent) }\SpecialCharTok{+}
  \FunctionTok{facet\_wrap}\NormalTok{(}\SpecialCharTok{\textasciitilde{}}\NormalTok{age\_cat)}
\end{Highlighting}
\end{Shaded}

\begin{Shaded}
\begin{Highlighting}[]
\CommentTok{\# Extended contingency table with conditional probabilities by age group}
\NormalTok{Whickham }\SpecialCharTok{\%\textgreater{}\%}
  \FunctionTok{count}\NormalTok{(smoker, age\_cat, outcome) }\SpecialCharTok{\%\textgreater{}\%}
  \FunctionTok{group\_by}\NormalTok{(smoker, age\_cat) }\SpecialCharTok{\%\textgreater{}\%}
  \FunctionTok{mutate}\NormalTok{(}\AttributeTok{prop =}\NormalTok{ n }\SpecialCharTok{/} \FunctionTok{sum}\NormalTok{(n)) }\SpecialCharTok{\%\textgreater{}\%}
  \FunctionTok{filter}\NormalTok{(outcome }\SpecialCharTok{==} \StringTok{"dead"}\NormalTok{) }\SpecialCharTok{\%\textgreater{}\%}
  \FunctionTok{select}\NormalTok{(smoker, age\_cat, prop) }\SpecialCharTok{\%\textgreater{}\%}
  \FunctionTok{pivot\_wider}\NormalTok{(}\AttributeTok{names\_from =}\NormalTok{ smoker, }\AttributeTok{values\_from =}\NormalTok{ prop)}
\end{Highlighting}
\end{Shaded}

What changed: When we control for age, the relationship reverses! Within
each age group, smokers now show higher mortality rates than
non-smokers, which aligns with our expectations.

What explains this change: This is Simpson's Paradox. The original
relationship was confounded by age because: 1. Younger people are more
likely to smoke 2. Younger people are also much less likely to die
during the 20-year study period 3. When we don't control for age, the
protective effect of being young masks the harmful effect of smoking

The age-stratified analysis reveals the true relationship: smoking
increases mortality risk within each age group.

Knit, \emph{commit, and push your changes to GitHub with an appropriate
commit message. Make sure to commit and push all changed files so that
your Git pane is cleared up afterwards and review the md document on
GitHub to make sure you're happy with the final state of your work.}

Now go back through your write up to make sure you've answered all
questions and all of your R chunks are properly labeled. Once you decide
that you are done with the homework, choose the knit drop down and
select \texttt{Knit\ to\ tufte\_handout} to generate a pdf. Download and
submit that pdf to Canvas.

\end{document}
